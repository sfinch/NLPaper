\section{Simple Analysis}

% TODO BETTER TITLE

An Elementary Cellular Automaton with $n$ sites per time step has
$n$ degrees of freedom.
As one studies ECAs with larger $n$ values, it becomes a daunting
task to track the state of each individual site at each time step, and
an analysis of the time evolution of the system as a whole becomes
more meaningful.
In order to study the general behavior of the ECA systems, we wanted
to define an observable which collapsed these $n$ degrees of freedom
into one number whose value could be tracked over many time steps.
However, there are only so many such observables for such simple
systems.
One ought to choose an observable whose value is meaningful for a wide
variety of initial conditions and boundary condition constraints.


\subsection{Defining an Observable}

We chose to study the density of black blocks as a convenient
% TODO collapse
observable for collapsing the information encoded in an $n$-site ECA
into one value per time step.
For ECAs with periodic or fixed boundary conditions, the density of
black blocks is straightforward:

\begin{equation}
    \rho = \frac{n_{black}}{n},
\end{equation}

\noindent where $n$ is the number of sites in the ECA.

One must be careful in the case of ECAs without boundary conditions
because the extent of the ECA is essentially infinite, and the above
definition of the black block density is not well-defined.
However, because the ECA rules only allow the state of a block to
depend on the previous states of its immediate neighbors, we chose
an initial condition of one black block surrounded by an infinite
number of white blocks on either side and constrainedourselves to
rules which allow black blocks to spread by only one site per time
step.
Under these conditions there can be no black blocks outside the sites
inside the triangle defined by $2t+1$, where $t$ is the number of time
steps since the initial state.
Treating $2t+1$ as the effective size of the Cellular Automaton at a
given time, the density of black blocks can be redefined as

\begin{equation}
    \rho = \frac{n_{black}}{(2t+1)}.
\end{equation}

% TODO
This observable has proved to be 


\subsection{Density Time Evolution}
% TODO MAXIMUM
For ECAs with periodic or fixed boundary conditions, there is a
maximum number of global states available to the system: if there are
$n$ sites and each must have a value of 0 or 1, there will be $2^n$
possible states.
As such, after at most $2^n$ time steps the system must return to
a state that it has visited previously, and of course the density as a
function of time will be periodic.

However, ECAs with no boundaries have access to an infinite number of
states and therefore can exhibit a much richer density spectrum.
Any observed periodic behavior is significant because the system is
not confined to a finite number of states, and any periodicity in the
density can only come about due to the intrinsic properties of a
particular rule.

Clearly not all ECA rules are created equally, and several of them are
not suitable for a study such as this 
For 

 - FFT and Autocorrelation of density for different rules

 - Periodic Boundary Conditions
