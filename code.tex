\newpage
\appendix
\section{Code}

\subsection{Density and FFT in \sc{Mathematica}}
\small
\begin{verbatim}

n = 1024;
m = CellularAutomaton[126, {{1}, 0}, n];
sumBlacks = Total[m, {2}]; 
Do[
  sumBlacks[[i]] = sumBlacks[[i]]/(2*(i - 1) + 1)
  , {i, 1, n + 1}
]

ArrayPlot[m, Frame->False]
ListLinePlot[sumBlacks, PlotRange -> All, 
  AxesLabel -> {"Time step", "Density"}]
ListLinePlot[Abs[Take[Fourier[sumBlacks], 150]],  PlotRange -> {0, 1},
  InterpolationOrder -> 2, AxesLabel -> {"Time steps", "FFT"}]

\end{verbatim}
\normalsize

\subsection{Mapping with PBC in \sc{C++}}
\small
\begin{verbatim}

// PBC.cxx
#include <stdlib.h>
#include <iostream>
#include <vector>

using namespace std;

// converts a long integer to a vector of booleans.
void LongToVec(const long &x, vector<bool> &v) {

    // make sure v is the right size.
    // this should not be time consuming if it started off with the
    // correct size.
    v.resize(sizeof(x)*4);

    long window = 1;
    for (int i = 0; i < v.size(); i++) {
        // set bit i to true if x's ith bit is also true.
        v[i] = x & window;

        // move the window left one bit.
        window <<= 1;
    }
    return;
}

// converts a vector of booleans to a long integer.
long VecToLong(const vector<bool> &v) {

    long x = 0;
    long window = 1;
    for (int i = 0; i < v.size(); i++) {
        // set x bit i to true if v's ith bit is also true.
        x |= v[i] ? window: 0;

        // move the window left one bit.
        window <<= 1;
    }
    return x;
}

bool ApplyRule(char rule, char prev) {
    return rule & (0x001 << prev);
}

long GetMap(char rule, long input, int nmax) {
    // initialize the vectors.
    vector<bool> v, vout;
    LongToVec(input, v);
    vout.resize(v.size(), false);

    char prev;
    for (int i = 1; i < nmax-1; i++) {
        // put the three neighboring blocks from the previous
        // iteration into a char for rule-applying.
        prev = (v[i+1] << 2) | (v[i] << 1) | v[i-1];

        // apply the rule and store the resulting bit in vout.
        vout[i] = ApplyRule(rule, prev);
    }

    // deal with i = 0 case.
    prev = (v[1] << 2) | (v[0] << 1) | v[nmax-1];
    vout[0] = ApplyRule(rule, prev);

    // deal with i = nmax-1 case.
    prev = (v[0] << 2) | (v[nmax-1] << 1) | v[nmax-2];
    vout[nmax-1] = ApplyRule(rule, prev);

    // convert vout back to a long integer.
    return VecToLong(vout);
}

int main(int argc, char *argv[]) {
    char rule = atoi(argv[1]);
    int n = atoi(argv[2]);

    int max = 1 << n;
    for (int i = 0; i < max; i++)
        cout << i << ", " << GetMap(rule, i, n) << endl;
    return 0;
}

\end{verbatim}
\normalsize
